\documentclass{elsarticle}

% Packages ====================================================================
\usepackage[utf8]{inputenc}
\usepackage{amsmath}
\usepackage{amssymb}

% Package set-up ==============================================================

% Article info ================================================================ 
\author[1]{Juan Valencia}
\ead{jmvalenciae@unal.edu.co}
\title{Some strange Title}

\begin{document}
\begin{keyword}
    MRI \sep CNNs
\end{keyword}
\begin{abstract}
    Elit possimus deserunt at minima facilis, qui ea. Obcaecati nobis facilis at
    corrupti optio Ea mollitia earum similique necessitatibus adipisci,
    obcaecati. Esse esse suscipit ipsa id totam Quia rerum praesentium debitis
    illum consequatur et? Eum fuga fugit provident fuga eum cumque, ullam
    tempore quaerat Repellendus commodi dolor repellendus ad illum
\end{abstract}
\maketitle
\section{Introduction}
Elit doloribus quod nulla minima hic, impedit? Praesentium velit minus soluta ut
ipsum ab. Nam maiores distinctio rerum laboriosam sit Tempora esse sapiente
tempore corporis iste nam ullam. Blanditiis suscipit consequuntur rerum
repellendus delectus! Optio reprehenderit omnis numquam perspiciatis fugit.
Asperiores illum reiciendis facilis dicta laboriosam Officiis laboriosam dolorem
nisi dolorum minima Eaque aperiam impedit quasi quod animi Molestiae quae ipsum
quas saepe maiores, minus molestiae Voluptatum nisi magni nesciunt quod dicta?
Quisquam quod expedita in totam enim? Eum quaerat inventore tenetur nobis quidem
nostrum voluptas Modi voluptate quia dolores placeat laborum repellendus, magnam
minus, quod. Enim modi quod pariatur?\par

\section{Materials and Methods}

For a 3D \emph{mri} image input $X_n = \{x_n^{s c a} \in \mathbb{R} | s \land c
\land a \in \mathbb{N}\}$ where  $s < S$, $c < C $ and $a < A$ \footnote{$S$,
$C$ and $A$ represents the number of \emph{sagittal}, \emph{coronal} and
\emph{axial} slices respectively};  we want to label each of its pixels to get a
labeled image output $Y_n = \{y_n^{sca} \in \{0, 1\}\}$. The image is obtained
using a model $f : \mathbb{R}^{S \times C \times A} \mapsto \mathbb{R}^{S \times
C \times A} $.
\begin{equation}
    \hat{Y}_n = f(X_n) = (\Psi_3 \circ \Psi_2 \circ \Psi_1) (X_n)
\end{equation}
$\Psi_1$, $\Psi_2$ and $\Psi_3$ correspond to the model segmentation, cropping
and correction steps respectively.\par


\subsection{CNNs}
The Segmentation and correction steps use the same approach, which consists of
passing $X_n$ in three different axial views through a model $\Phi$, which is a
sequence of residual block layers. For Example $X_n$ can be represented as a set
of sagittal slices $X_n = \{x_n^s \in \mathcal{R}^{C \times A}\}_{s=1}^S$ and we
want to get a set of sagittal slices $Y_n = \{y_n^s \in [0, 1]^{C \times
A}\}_{s=1}^S$:
\begin{equation}
    Y_n = \Phi(X_n) = (\phi_L \circ \cdots \circ \phi_l \circ \cdots \circ
    \phi_1)(X_n)
\end{equation}
where $\phi_l$ is a residual block layer of the model $\Phi$.\par

$\Phi$ model performance depends of $\omega$ a set of parameters which must be
optimized as follows:
\begin{equation}
    \omega^* = \arg \min_{\omega} \mathcal{L}(Y_n, \Phi(X_n))
\end{equation}
being $\mathcal{L}(Y_n, \Phi(X_n)) = \frac{1}{|Y_n|} \| Y_n - \Phi(X_n)\|_F^2$ 

\section{Implementation}
\section{Results and Discussion}
\section{Conclusions}
\end{document}
